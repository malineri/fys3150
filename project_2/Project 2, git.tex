\documentclass[english,notitlepage]{revtex4-1}  % defines the basic parameters of the document
%For preview: skriv i terminal: latexmk -pdf -pvc filnavn



% if you want a single-column, remove reprint

% allows special characters (including æøå)
\usepackage[utf8]{inputenc}
%\usepackage[english]{babel}

%% note that you may need to download some of these packages manually, it depends on your setup.
%% I recommend downloading TeXMaker, because it includes a large library of the most common packages.

\usepackage{physics,amssymb}  % mathematical symbols (physics imports amsmath)
\include{amsmath}
\usepackage{graphicx}         % include graphics such as plots
\usepackage{xcolor}           % set colors
\usepackage{hyperref}         % automagic cross-referencing (this is GODLIKE)
\usepackage{listings}         % display code
\usepackage{subfigure}        % imports a lot of cool and useful figure commands
\usepackage{float}
%\usepackage[section]{placeins}
\usepackage{algorithm}
\usepackage[noend]{algpseudocode}
\usepackage{subfigure}
\usepackage{tikz}
\usetikzlibrary{quantikz}
% defines the color of hyperref objects
% Blending two colors:  blue!80!black  =  80% blue and 20% black
\hypersetup{ % this is just my personal choice, feel free to change things
    colorlinks,
    linkcolor={red!50!black},
    citecolor={blue!50!black},
    urlcolor={blue!80!black}}

%% Defines the style of the programming listing
%% This is actually my personal template, go ahead and change stuff if you want



%% USEFUL LINKS:
%%
%%   UiO LaTeX guides:        https://www.mn.uio.no/ifi/tjenester/it/hjelp/latex/
%%   mathematics:             https://en.wikibooks.org/wiki/LaTeX/Mathematics

%%   PHYSICS !                https://mirror.hmc.edu/ctan/macros/latex/contrib/physics/physics.pdf

%%   the basics of Tikz:       https://en.wikibooks.org/wiki/LaTeX/PGF/Tikz
%%   all the colors!:          https://en.wikibooks.org/wiki/LaTeX/Colors
%%   how to draw tables:       https://en.wikibooks.org/wiki/LaTeX/Tables
%%   code listing styles:      https://en.wikibooks.org/wiki/LaTeX/Source_Code_Listings
%%   \includegraphics          https://en.wikibooks.org/wiki/LaTeX/Importing_Graphics
%%   learn more about figures  https://en.wikibooks.org/wiki/LaTeX/Floats,_Figures_and_Captions
%%   automagic bibliography:   https://en.wikibooks.org/wiki/LaTeX/Bibliography_Management  (this one is kinda difficult the first time)
%%   REVTeX Guide:             http://www.physics.csbsju.edu/370/papers/Journal_Style_Manuals/auguide4-1.pdf
%%
%%   (this document is of class "revtex4-1", the REVTeX Guide explains how the class works)


%% CREATING THE .pdf FILE USING LINUX IN THE TERMINAL
%%
%% [terminal]$ pdflatex template.tex
%%
%% Run the command twice, always.
%% If you want to use \footnote, you need to run these commands (IN THIS SPECIFIC ORDER)
%%
%% [terminal]$ pdflatex template.tex
%% [terminal]$ bibtex template
%% [terminal]$ pdflatex template.tex
%% [terminal]$ pdflatex template.tex
%%
%% Don't ask me why, I don't know.

\begin{document}

\title{Project 2}      % self-explanatory
\author{Malin Eriksen}          % self-explanatory
\date{\today}                             % self-explanatory
\noaffiliation                            % ignore this, but keep it.


\maketitle 
    
\textit{https://github.com/malineri/fys3150/tree/main/project\_2}
    
\section*{Problem 1}
In this project we will be looking at a horizontal beam. The beam will be of length L, and a force F will be applied in the endpoint of this beam. This will be described by the second-order differential equation (1).
\begin{equation}\label{eq:newton}
    \gamma \frac{d^2 u(x)}{dx^2} = -F u(x).
\end{equation}
We begin with scaling this formula into a dimensionless equation, by changing the x to a unitless variable $\hat{x} = \frac{x}{L}$. The derivation term then becomes, 
\begin{align*}
\frac{d}{dx} = \frac{d\hat{x}}{dx} \frac{d}{d\hat{x}} = \frac{1}{L} \frac{d}{d \hat{x}}.
\end{align*}
We have a second order differential equation, so we have to do this operation twice. 
Adding this to equation(1) we get the formula,
\begin{align*}
 \frac{\gamma}{L^2} \frac{d^2 u(\hat{x})}{d\hat{x}^2} = -F u(\hat{x}).
\end{align*}
Which we rearrange slightly, 
\begin{align*}
\frac{d^2 u(\hat{x})}{d\hat{x}^2} = - \frac{F L^2}{\gamma} u(\hat{x}).
\end{align*}
Now we introduce a new variable lambda $\lambda = \frac{F L^2}{\gamma}$ then we find that the dimensionless equation of our first equation can be written as, 
\begin{equation}\label{eq:newton}
    \frac{d^2 u(\hat{x})}{d\hat{x}^2} = - \lambda u(\hat{x}).
\end{equation} 
\newpage{}



\section*{Problem 2}\
\\
To solve this problem we will be using matrices. We write a short program to set up a tridiagonal NxN matrix A, when N = 6. We want this matrix to solve the classic eigenvector, eigenvalue problem $\mathbf{A} \vec{v} = \lambda \vec{v}$ in the same way we formulated our dimensionless equation(2).\footnote{https://stackoverflow.com/questions/70556590/how-to-solve-for-eigen-value-of-a-matrix-numerically} We being with creating an algorithm with a general function that determines a tridiagonal matrix. This algorithm we see in Algorithm 1.
\begin{algorithm}[H]
    \caption{Creating function that takes values from diagonal and makes a tridiagonal matrix.}\label{av_lik_lv}
    \begin{algorithmic}
    	\State int main(){
	\State arma::mat create tridiagonal(int n, double a, double d, double e)
	\State arma::mat A = arma::mat(n, n, arma::fill::eye); \Comment{n x n identity matrix}
        \State int N = 6  \Comment{We begin with setting the length of the matrix N = 6}
        \State A(0, 0) = d; A(1, 0) = e; A(0, 1) = a;  \Comment{Manually filling in values of A, Could potentially be done more efficiantly in a loop.}
 	\State return A;
	\State }
    \end{algorithmic}
\end{algorithm}
We now want to put in values of N, a, d and e, to determine our exact values. We want those to be a, $e = -1/h^2$ and $d = -2/h^2$. And we also want the program to print the eigenvalues and eigenvectors. A program returning the matrix, its eigenvalue and its corresponding eigenvectors is written in Algorithm 2.
\begin{algorithm}[H]
    \caption{Our main containing the values of our matrix, and printing the matrix, its eigenvector and eigenvalues.}\label{av_lik_lv}
    \begin{algorithmic}
    	\State int main(){
	\State int N = 6.; float h = 1.; float a = (-1.)/(h*h); float d = (2.)/(h*h); \Comment{Setting values we use in the matrix}
	\State arma::mat A = create tridiagonal(N, a, d, a); \Comment{Create matrix A from function}
        \State int N = 6  \Comment{We begin with setting the length of the matrix N = 6}
        \State arma::vec eigval;
        \State arma::mat eigvec;
        \State arma::eig sym(eigval, eigvec, A);
        \State int width = 18; int prec = 10; \Comment{Parameters for output formatting}
        \State std::cout $<<$ ”\#” $<<$ std::setw(width-1) $<<$ A
        \State $<<$ std::endl;
        \State std::cout $<<$ ”\#” $<<$ std::setw(width-1) $<<$ eigvec
        \State $<<$ std::endl;
        \State std::cout $<<$ ”\#” $<<$ std::setw(width-1) $<<$ eigval
        \State $<<$ std::endl;
 	\State return 0;
	\State }
    \end{algorithmic}
\end{algorithm}
Now that we have found the eigenvalues and corresponding eigenvectors we print the results, to make it easier to compare the analytical and numerical solutions we remove the /h2 part of the a, d and e values, so that we use the simple tridiagonal matrix with 2 on the diagonal and -1 on the upper and lower diagonal. This gives us the results we can see in table(1).
\begin{table}%[h!]
    \centering
    \caption{Eigenvalues and corresponding eigenvectors of a tridiagonal matrix A}
    \begin{tabular}{c@{\hspace{1cm}} c}
        \hline
        Eigenvalues $\lambda$ & Eigenvectors $\vec{v}$ \\
        \hline
        0.1981 & [0.2319, 0.4179, 0.5211, 0.5211, 0.4179, 0.2319] \\
        0.7530 &  [-0.4179, -0.5211, -0.2319, 0.2319, 0.5211, 0.4179] \\
        1.5550 & [0.5211, 0.2319, -0.4179, -0.4179, 0.2319, 0.5211]\\
        2.4450 &  [ 0.5211, -0.2319, -0.4179, 0.4179, 0.2319, -0.5211]\\
        3.2470 &  [0.4179, -0.5211, 0.2319, 0.2319, -0.5211, 0.4179]\\
        3.8019 & [-0.2319, 0.4179, -0.5211, 0.5211, -0.4179, 0.2319]\\
        \hline
    \end{tabular}\label{tab:output_table}
\end{table}
We can check if these values are correct by comparing the results to the analytical results with the formulas
\begin{align*}
\lambda^{i} = d + 2a \cos \bigg(\frac{i \pi}{N + 1} \bigg) 
\end{align*}
\begin{align*}
\vec{v}^{i} = \bigg[ \sin \bigg(\frac{i \pi}{N + 1} \bigg), \sin \bigg( \frac{2i \pi}{N + 1} \bigg), ... , \sin \bigg(\frac{Ni \pi}{N + 1} \bigg) \bigg]^T 
\end{align*}
Where i = 1, ... , N. 
We make a program that solves this algorithm for our N = 6 case. This program is called \textit{analytical 2.py}, and the algorithm is added to the github repo. We get the same values on the eigenvalues, but we get different eigenvectors, something that tells us there must be a small mistake in one of the programs. If time I will go back and look at potential mistakes in the program. In table(2) we see the values we get from the analytical solution.
\begin{table}%[h!]
    \centering
    \caption{Eigenvalues and corresponding eigenvectors of a tridiagonal matrix A}
    \begin{tabular}{c@{\hspace{1cm}} c}
        \hline
        Eigenvalues $\lambda$ & Eigenvectors $\vec{v}$ \\
        \hline
        0.1981 & [0.43388374, -0.78183148, 0.97492791, -0.97492791,0.78183148, -0.43388374]\\
        0.7530 &  [0.78183148, -0.97492791, 0.43388374, 0.43388374, -0.97492791, 0.78183148]\\
        1.5550 & [0.97492791, -0.43388374, -0.78183148, 0.78183148, 0.43388374, -0.97492791]\\
        2.4450 &  [0.97492791, 0.43388374, -0.78183148, -0.78183148, 0.43388374, 0.97492791]\\
        3.2470 &  [0.78183148, 0.97492791, 0.43388374, -0.43388374, -0.97492791, -0.78183148]\\
        3.8019 & [0.43388374, 0.78183148, 0.97492791, 0.97492791, 0.78183148, 0.43388374]\\
        \hline
    \end{tabular}\label{tab:output_table}
\end{table}

 



\section*{Problem 3}
-
   
\end{document}
