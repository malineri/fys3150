\documentclass[english,notitlepage]{revtex4-1}  % defines the basic parameters of the document
%For preview: skriv i terminal: latexmk -pdf -pvc filnavn



% if you want a single-column, remove reprint

% allows special characters (including æøå)
\usepackage[utf8]{inputenc}
%\usepackage[english]{babel}

%% note that you may need to download some of these packages manually, it depends on your setup.
%% I recommend downloading TeXMaker, because it includes a large library of the most common packages.

\usepackage{physics,amssymb}  % mathematical symbols (physics imports amsmath)
\include{amsmath}
\usepackage{graphicx}         % include graphics such as plots
\usepackage{xcolor}           % set colors
\usepackage{hyperref}         % automagic cross-referencing (this is GODLIKE)
\usepackage{listings}         % display code
\usepackage{subfigure}        % imports a lot of cool and useful figure commands
\usepackage{float}
%\usepackage[section]{placeins}
\usepackage{algorithm}
\usepackage[noend]{algpseudocode}
\usepackage{subfigure}
\usepackage{tikz}
\usetikzlibrary{quantikz}
% defines the color of hyperref objects
% Blending two colors:  blue!80!black  =  80% blue and 20% black
\hypersetup{ % this is just my personal choice, feel free to change things
    colorlinks,
    linkcolor={red!50!black},
    citecolor={blue!50!black},
    urlcolor={blue!80!black}}

%% Defines the style of the programming listing
%% This is actually my personal template, go ahead and change stuff if you want



%% USEFUL LINKS:
%%
%%   UiO LaTeX guides:        https://www.mn.uio.no/ifi/tjenester/it/hjelp/latex/
%%   mathematics:             https://en.wikibooks.org/wiki/LaTeX/Mathematics

%%   PHYSICS !                https://mirror.hmc.edu/ctan/macros/latex/contrib/physics/physics.pdf

%%   the basics of Tikz:       https://en.wikibooks.org/wiki/LaTeX/PGF/Tikz
%%   all the colors!:          https://en.wikibooks.org/wiki/LaTeX/Colors
%%   how to draw tables:       https://en.wikibooks.org/wiki/LaTeX/Tables
%%   code listing styles:      https://en.wikibooks.org/wiki/LaTeX/Source_Code_Listings
%%   \includegraphics          https://en.wikibooks.org/wiki/LaTeX/Importing_Graphics
%%   learn more about figures  https://en.wikibooks.org/wiki/LaTeX/Floats,_Figures_and_Captions
%%   automagic bibliography:   https://en.wikibooks.org/wiki/LaTeX/Bibliography_Management  (this one is kinda difficult the first time)
%%   REVTeX Guide:             http://www.physics.csbsju.edu/370/papers/Journal_Style_Manuals/auguide4-1.pdf
%%
%%   (this document is of class "revtex4-1", the REVTeX Guide explains how the class works)


%% CREATING THE .pdf FILE USING LINUX IN THE TERMINAL
%%
%% [terminal]$ pdflatex template.tex
%%
%% Run the command twice, always.
%% If you want to use \footnote, you need to run these commands (IN THIS SPECIFIC ORDER)
%%
%% [terminal]$ pdflatex template.tex
%% [terminal]$ bibtex template
%% [terminal]$ pdflatex template.tex
%% [terminal]$ pdflatex template.tex
%%
%% Don't ask me why, I don't know.

\begin{document}

\title{Project 2}      % self-explanatory
\author{Malin Eriksen}          % self-explanatory
\date{\today}                             % self-explanatory
\noaffiliation                            % ignore this, but keep it.


\maketitle 
    
\textit{https://github.com/malineri/fys3150/tree/main/project\_2}
    
\section*{Problem 1}
In this project we will be looking at a horizontal beam. The beam will be of length L, and a force F will be applied in the endpoint of this beam. This will be described by the second-order differential equation (1).
\begin{equation}\label{eq:newton}
    \gamma \frac{d^2 u(x)}{dx^2} = -F u(x).
\end{equation}
We begin with scaling this formula into a dimensionless equation, by changing the x to a unitless variable $\hat{x} = \frac{x}{L}$. The derivation term then becomes, 
\begin{align*}
\frac{d}{dx} = \frac{d\hat{x}}{dx} \frac{d}{d\hat{x}} = \frac{1}{L} \frac{d}{d \hat{x}}.
\end{align*}
We have a second order differential equation, so we have to do this operation twice. 
Adding this to equation(1) we get the formula,
\begin{align*}
 \frac{\gamma}{L^2} \frac{d^2 u(\hat{x})}{d\hat{x}^2} = -F u(\hat{x}).
\end{align*}
Which we rearrange slightly, 
\begin{align*}
\frac{d^2 u(\hat{x})}{d\hat{x}^2} = - \frac{F L^2}{\gamma} u(\hat{x}).
\end{align*}
Now we introduce a new variable lambda $\lambda = \frac{F L^2}{\gamma}$ then we find that the dimensionless equation of our first equation can be written as, 
\begin{equation}\label{eq:newton}
    \frac{d^2 u(\hat{x})}{d\hat{x}^2} = - \lambda u(\hat{x}).
\end{equation} 
\newpage{}



\section*{Problem 2}
To solve this problem we will be using matrices. We write a short program to set up a tridiagonal NxN matrix \textbf{A}, when N = 6. We want this matrix to solve the classic eigenvector, eigenvalue problem $\mathbf{A}\vec{v} = \lambda \vec{v}$ in the same way we formulated our dimensionless equation(2).\footnote{https://stackoverflow.com/questions/70556590/how-to-solve-for-eigen-value-of-a-matrix-numerically}
\begin{algorithm}[H]
    \caption{Set up for matrix equation of eigenvalues and eigenvectors in Python}\label{av_lik_lv}
    \begin{algorithmic}
    	%\State int main()
	\State from scipy.sparse import coo\_matrix
	\State import numpy as np
        \State int N = 6  \Comment{We begin with setting the length of the matrix N = 6}
        \State a, b, c = [1]*(N - 1), [-2]*N, [1]*(N - 1);  \Comment{Creating a NxN tridiagonal matrix, with 1, -2 and 1.}
 	\State A = np.diag(a, -1) + np.diag(b, 0) + np.diag(c, 1)
        \State l, v = np.linalg.eig(A)  \Comment{Finding eigenvalues $\lambda$ and eigenvektors $\vec{v}$}
        \State assert np.allclose(v @ np.diag(l), A @ v)
    \end{algorithmic}
\end{algorithm}
Now that we have found the eigenvalues and corresponding eigenvectors we print the results, which we can see in table (1). 
\begin{table}%[h!]
    \centering
    \caption{Eigenvalues and corresponding eigenvectors of a tridiagonal matrix A}
    \begin{tabular}{c@{\hspace{1cm}} c}
        \hline
        Eigenvalues $\lambda$ & Eigenvectors $\vec{v}$ \\
        \hline
       $-3.80193774$ & [-0.23192061, -0.41790651, -0.52112089, -0.23192061, 0.52112089, -0.41790651]\\
        $-3.2469796$ &  [0.41790651,  0.52112089,  0.23192061, -0.41790651,  0.23192061, -0.52112089] \\
        $-2.44504187$ & [-0.52112089, -0.23192061,  0.41790651, -0.52112089, -0.41790651, -0.23192061]\\
        $-0.19806226$ &  [ 0.52112089, -0.23192061, -0.41790651, -0.52112089, -0.41790651,  0.23192061]\\
        $-1.55495813$ &  [-0.41790651,  0.52112089, -0.23192061, -0.41790651,  0.23192061,  0.52112089]\\
        $-0.7530204$ & [ 0.23192061, -0.41790651,  0.52112089, -0.23192061,  0.52112089,  0.41790651]\\
        \hline
    \end{tabular}\label{tab:output_table}
\end{table}
We can check if these values are correct by comparing the results to the analytical results with the formulas
\begin{align*}
\lambda^{i} = d + 2a \cos \bigg(\frac{i \pi}{N + 1} \bigg) 
\end{align*}
\begin{align*}
\vec{v}^{i} = \bigg[ \sin \bigg(\frac{i \pi}{N + 1} \bigg), \sin \bigg( \frac{2i \pi}{N + 1} \bigg), ... , \sin \bigg(\frac{Ni \pi}{N + 1} \bigg) \bigg]^T 
\end{align*}
Where i = 1, ... , N. 





In this program we used the matrix A, which is just the second derivative, but without the special values of our case. 
\begin{align*}
     A = \begin{bmatrix}
    2 & 1 & 0 & 0 & 0 & 0 \\
    1 & 2 & 1 & 0 & 0 & 0 \\
    0 & 1 & 2 & 1 & 0 & 0 \\
    0 & 0 & 1 & 2 & 1 & 0 \\
    0 & 0 & 0 & 1 & 2 & 1 \\
    0 & 0 & 0 & 0 & 1 & 2
    \end{bmatrix}
\end{align*}


 



\section*{Problem 3}
-
   
\end{document}
